\setstretch{1.2}%řádkování
\hspace{1 cm}
Pro kopii: \texttt{git clone \href{https://github.com/kovarmi9/PJIN\_mobilni\_bluetooth\_aplikace}{https://github.com/kovarmi9/PJIN\_mobilni\_bluetooth\_aplikace}.git}.
Pro spuštění kódu aplikace je potřeba mít nainstalovaný \textbf{Node.js} a \textbf{npm}. \textbf{Node.js} je prostředí, které umožňuje spouštění \textbf{JavaScriptu(TypeScriptu)} mimo prohlížeč. \textbf{Npm (Node Package Manager)} je balíčkovací systém pro \textbf{JavaScript}, který je součástí instalace \textbf{Node.js}. \textbf{Node.js} a \textbf{npm} lze stáhnout ze stránek:\href{https://nodejs.org/en}{https://nodejs.org/en}. Dále bude potřeba \textbf{React Native CLI}. \textbf{React Native CLI} je nástroj, který umožňuje vytvářet a spravovat projekty \textbf{React Native}. Lze jej nainstalovat pomocí \textbf{npm} příkazem \texttt{npm install -g react-native-cli}. 
Po instalaci těchto nástrojů je potřeba ještě zařízení na němž bude probíhat ladění kódu aplikace. K tomu může posloužit virtuální zařízení. Virtuální zařízení \textbf{Android} zle vytvořit s pomocí \textbf{Android Studia} nebo pro tvorbu virtuálního zařízení s \textbf{iOS} bude potřeba \textbf{Mac} a software \textbf{Xcode}. Virtuální zařízení ovšem nepodporují práci s Bluetooth a tak je lepší alternativou pro ladění aplikace využít skutečné mobilní zařízení s Androidem. Nejprve je nutné na mobilním zařízení povolit možnosti pro vývojáře. To se provede v \texttt{nastavení} \(\rightarrow\) \texttt{o telefonu} \(\rightarrow\) \texttt{informace o software} \(\rightarrow\) \texttt{číslo sestavení} a po té co je sedmkrát poklepáno na možnost \texttt{číslo sestavení} se otevřou možnosti pro vývojáře. Dále je nutné v možnostech pro vývojáře povolit možnost \texttt{Ladění USB}. Po té co je toto nastavení provedeno, je mobilní zařízení připojeno k počítači prostřednictvím USB kabelu a je v adresáři s projektem v terminálu spuštěn příkaz \texttt{npm start} a zvolena možnost \texttt{a} pro Android se na mobilním zařízení vytvoří prostředí aplikace, které se přizpůsobuje změnám kódu. Před prvním spuštěním aplikace je ještě potřeba do terminálu v adresáři s projektem zadat příkaz \texttt{npm install} který by měl automaticky nainstalovat všechny balíčky, které jsou využívány v projektu.